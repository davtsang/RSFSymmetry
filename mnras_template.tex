% mnras_template.tex 
%
% LaTeX template for creating an MNRAS paper
%
% v3.0 released 14 May 2015
% (version numbers match those of mnras.cls)
%
% Copyright (C) Royal Astronomical Society 2015
% Authors:
% Keith T. Smith (Royal Astronomical Society)

% Change log
%
% v3.0 May 2015
%    Renamed to match the new package name
%    Version number matches mnras.cls
%    A few minor tweaks to wording
% v1.0 September 2013
%    Beta testing only - never publicly released
%    First version: a simple (ish) template for creating an MNRAS paper

%%%%%%%%%%%%%%%%%%%%%%%%%%%%%%%%%%%%%%%%%%%%%%%%%%
% Basic setup. Most papers should leave these options alone.
\documentclass[fleqn,usenatbib]{mnras}

% MNRAS is set in Times font. If you don't have this installed (most LaTeX
% installations will be fine) or prefer the old Computer Modern fonts, comment
% out the following line
\usepackage{newtxtext,newtxmath}
% Depending on your LaTeX fonts installation, you might get better results with one of these:
%\usepackage{mathptmx}
%\usepackage{txfonts}

% Use vector fonts, so it zooms properly in on-screen viewing software
% Don't change these lines unless you know what you are doing
\usepackage[T1]{fontenc}

% Allow "Thomas van Noord" and "Simon de Laguarde" and alike to be sorted by "N" and "L" etc. in the bibliography.
% Write the name in the bibliography as "\VAN{Noord}{Van}{van} Noord, Thomas"
\DeclareRobustCommand{\VAN}[3]{#2}
\let\VANthebibliography\thebibliography
\def\thebibliography{\DeclareRobustCommand{\VAN}[3]{##3}\VANthebibliography}


%%%%% AUTHORS - PLACE YOUR OWN PACKAGES HERE %%%%%

% Only include extra packages if you really need them. Common packages are:
\usepackage{graphicx}	% Including figure files
\usepackage{amsmath}	% Advanced maths commands
\usepackage{amssymb}	% Extra maths symbols

%%%%%%%%%%%%%%%%%%%%%%%%%%%%%%%%%%%%%%%%%%%%%%%%%%

%%%%% AUTHORS - PLACE YOUR OWN COMMANDS HERE %%%%%

% Please keep new commands to a minimum, and use \newcommand not \def to avoid
% overwriting existing commands. Example:
%\newcommand{\pcm}{\,cm$^{-2}$}	% per cm-squared

%%%%%%%%%%%%%%%%%%%%%%%%%%%%%%%%%%%%%%%%%%%%%%%%%%

%%%%%%%%%%%%%%%%%%% TITLE PAGE %%%%%%%%%%%%%%%%%%%

% Title of the paper, and the short title which is used in the headers.
% Keep the title short and informative.
\title[Short title, max. 45 characters]{MNRAS \LaTeXe\ template -- title goes here}

% The list of authors, and the short list which is used in the headers.
% If you need two or more lines of authors, add an extra line using \newauthor
\author[D. Neill et al.]{
Duncan Neill,$^{1}$\thanks{E-mail: dn431@bath.ac.uk}
William Newton,$^{2}$
David Tsang$^{1}$
\\
% List of institutions
$^{1}$Department of Physics, University of Bath, Claverton Down, Bath, BA1 1AL\\
$^{2}$Department of Physics and Astronomy, Texas A\&M University-Commerce, Commerce, TX, 75429-3011
}

% These dates will be filled out by the publisher
\date{Accepted XXX. Received YYY; in original form ZZZ}

% Enter the current year, for the copyright statements etc.
\pubyear{2015}

% Don't change these lines
\begin{document}
\label{firstpage}
\pagerange{\pageref{firstpage}--\pageref{lastpage}}
\maketitle

% Abstract of the paper
\begin{abstract}
We do physics and calculate stuff. 
Such Physics. Many maths. Wow.
\end{abstract}

% Select between one and six entries from the list of approved keywords.
% Don't make up new ones.
\begin{keywords}
keyword1 -- keyword2 -- keyword3
\end{keywords}

%%%%%%%%%%%%%%%%%%%%%%%%%%%%%%%%%%%%%%%%%%%%%%%%%%

%%%%%%%%%%%%%%%%% BODY OF PAPER %%%%%%%%%%%%%%%%%%

% \section{Introduction}

% This is a simple template for authors to write new MNRAS papers.
% See \texttt{mnras\_sample.tex} for a more complex example, and \texttt{mnras\_guide.tex}
% for a full user guide.

% All papers should start with an Introduction section, which sets the work
% in context, cites relevant earlier studies in the field by \citet{Fournier1901},
% and describes the problem the authors aim to solve \citep[e.g.][]{vanDijk1902}.
% Multiple citations can be joined in a simple way like \citet{deLaguarde1903, delaGuarde1904}.

% \section{Methods, Observations, Simulations etc.}

% Normally the next section describes the techniques the authors used.
% It is frequently split into subsections, such as Section~\ref{sec:maths} below.

% \subsection{Maths}
% \label{sec:maths} % used for referring to this section from elsewhere

% Simple mathematics can be inserted into the flow of the text e.g. $2\times3=6$
% or $v=220$\,km\,s$^{-1}$, but more complicated expressions should be entered
% as a numbered equation:

% \begin{align}
%     x=\frac{-b\pm\sqrt{b^2-4ac}}{2a}.
% 	\label{eq:quadratic}
% \end{align}

% Refer back to them as e.g. equation~(\ref{eq:quadratic}).

% \subsection{Figures and tables}

% Figures and tables should be placed at logical positions in the text. Don't
% worry about the exact layout, which will be handled by the publishers.

% Figures are referred to as e.g. Fig.~\ref{fig:example_figure}, and tables as
% e.g. Table~\ref{tab:example_table}.

% % Example figure
% \begin{figure}
% 	% To include a figure from a file named example.*
% 	% Allowable file formats are eps or ps if compiling using latex
% 	% or pdf, png, jpg if compiling using pdflatex
% 	\includegraphics[width=\columnwidth]{example}
%     \caption{This is an example figure. Captions appear below each figure.
% 	Give enough detail for the reader to understand what they're looking at,
% 	but leave detailed discussion to the main body of the text.}
%     \label{fig:example_figure}
% \end{figure}

% % Example table
% \begin{table}
% 	\centering
% 	\caption{This is an example table. Captions appear above each table.
% 	Remember to define the quantities, symbols and units used.}
% 	\label{tab:example_table}
% 	\begin{tabular}{lccr} % four columns, alignment for each
% 		\hline
% 		A & B & C & D\\
% 		\hline
% 		1 & 2 & 3 & 4\\
% 		2 & 4 & 6 & 8\\
% 		3 & 5 & 7 & 9\\
% 		\hline
% 	\end{tabular}
% \end{table}


% \section{Conclusions}

% The last numbered section should briefly summarise what has been done, and describe
% the final conclusions which the authors draw from their work.

% \section*{Acknowledgements}

% The Acknowledgements section is not numbered. Here you can thank helpful
% colleagues, acknowledge funding agencies, telescopes and facilities used etc.
% Try to keep it short.

% %%%%%%%%%%%%%%%%%%%%%%%%%%%%%%%%%%%%%%%%%%%%%%%%%%
% \section*{Data Availability}

 
% The inclusion of a Data Availability Statement is a requirement for articles published in MNRAS. Data Availability Statements provide a standardised format for readers to understand the availability of data underlying the research results described in the article. The statement may refer to original data generated in the course of the study or to third-party data analysed in the article. The statement should describe and provide means of access, where possible, by linking to the data or providing the required accession numbers for the relevant databases or DOIs.

\section{Introduction}
\hspace{\parindent}Neutron stars consist of extremely dense matter, the study of which is important for our understanding of nuclear physics. To study the internal structure of these stars and the nuclear physics that determines it, we must find ways to probe them using observational phenomena. Short gamma-ray bursts (sGRBs)~\cite{d2015short} are one such phenomena that, while their origins are uncertain, likely originate from the merging of binary neutron stars. Around 10\% of sGRBs are preceded by a 'precursor' flare~\cite{troja2010precursors}. These precursor flares can be identified as a lower, separate peak in the gamma-ray count a short time ($\sim 0.1-1.0$ s) before the main peak~\cite{zhong2019precursors}. In this paper we will investigate the constraints on nuclear physics due to one possible cause of precursor flares.
%mention possible causes of precursors (here or later?)
%describe what a sGRB is / looks like?

\hspace{\parindent}In the last few years multi-messenger observations of neutron star mergers have provided insight into the origin of sGRBs. In particular, the detection of the gravitational wave signals GW170817~\cite{abbott2017merger} and GW190425~\cite{abbott2020gw190425} that are thought to originate from mergers has provided a new angle from which to investigate neutron stars. Here, we use the frequency of gravitational waves to investigate the possibility of a relationship between precursor flares and the modes of oscillation within a neutron star.

\hspace{\parindent}Within a star the bulk motion of matter can be depicted as a combination of an infinite set of oscillation modes~\cite{smeyers2011linear}, each of which obeys the boundary conditions for the displacement of matter at the surface and core of the star. Each mode can be defined by the frequency at which it oscillates and the displacement of matter it causes throughout the star. These oscillations can be driven by external gravitational fields, such as that of a binary partner. They are strongly dependant on the internal structure of the star, with different modes being dependant on different properties, such as pressure or shear forces.


%define core and crust (and ocean)? or assume reader knowns what they are.





\subsection{Resonant Shattering Flares}
\hspace{\parindent}During the in-spiral of binary neutron stars, the oscillation modes can become excited by the tidal forces from the binary partner. Each oscillation mode overlaps with the tidal field to a different degree, which determines that amount of energy deposited in that specific mode. As the stars' orbits get smaller, the frequency of the orbit (and hence the gravitational wave frequency) increases. When the orbital frequency matches a mode's frequency it may become resonantly excited, causing the oscillations to rapidly grow in magnitude~\cite{tsang2012resonant}~\cite{tsang2013shattering}.

\hspace{\parindent}The theory of resonant shattering flares is that if the overlap between the tidal field and an oscillation mode is great enough, the resonant excitation of that mode may cause the neutron star crust to become deformed~\cite{troja2010precursors}. If this deformation becomes large enough the strain in the crust may exceed its limit, causing it to fracture. These fractures deposit energy in the crust which could cause it to shatter, creating high frequency oscillations that couple to the magnetic field to release a precursor flare shortly before the main sGRB.%explain RSF process better, explain (why)/(what is) the interface mode

\hspace{\parindent}Using multi-messenger observations of neutron star mergers, it is possible that the gravitational wave frequency at which a precursor flare occurs could be found. In order for a mode to experience resonant excitation, the frequency of the gravitational waves (i.e. the orbital frequency) must be similar to the frequency of that mode. Therefore, by calculating the frequency of an oscillation mode that has a strong overlap with the tidal field using a variety of nuclear models, the nuclear parameters could be restricted to those which result in a mode frequency that closely matches the observed gravitational wave frequency during the precursor flare. 

\hspace{\parindent}Of particular interest to the RSF process is the interface (i) mode. This mode is strongest at the boundary between the core and the crust, and has a strong overlap with the tidal field~\cite{tsang2012resonant}. This makes it a good candidate for the source of RSFs as less time will be needed at resonance for it to reach the elastic limit for the crust to begin to fracture. In this work we will focus on the i-mode.








\subsection{Nuclear Symmetry Energy and Neutron Star Structure}
\hspace{\parindent}The oscillation frequencies of the modes are dependant on the neutron star equation of state (EoS), the relationship between the energy density and pressure inside the star. At low densities the EoS can be accurately calculated by using experimental data~\cite{baym1971ground}. However, at the extreme densities found in neutron stars the EoS is poorly understood and therefore we must rely on nuclear models. In order to reproduce physical results, these models can be parameterised by various nuclear properties, such as the nuclear symmetry energy parameters. The nuclear symmetry energy is the energy required to convert a proton into a neutron (or vice-versa) when the neutron and proton numbers are equal. At nuclear saturation density (the density of nucleons in a nucleus) the first three symmetry energy parameters are: the symmetry energy
\begin{align}
J=E_{\rm sym}(n_0),    
\label{eq:sym_J}
\end{align}
\noindent its first derivative 
\begin{align}
L=3n_0\frac{\partial E_{\rm sym}(n_b)}{\partial n_b}\biggr\rvert_{n_b=n_0},  
\label{eq:sym_L}
\end{align}
\noindent and its second derivative
\begin{align}
K_{\rm sym}=9n_0^2\frac{\partial^2 E_{\rm sym}(n_b)}{\partial n_b^2}\biggr\rvert_{n_b=n_0},
\label{eq:sym_K}
\end{align}
\noindent where $n_0\approx 0.16$ fm$^{-3}$ is the nuclear saturation density and $n_b$ is the baryon density.

\hspace{\parindent}To investigate the nuclear symmetry energy parameters we generated a set of EoSs parameterised by a reasonable range of values for $J$ ,$L$ and $K_{\rm sym}$. By calculating the mode frequencies for these EoSs we are able to find the relationship between the symmetry energy parameters and the gravitational wave frequency at which a RSF occurs.







\section{Parameterised Neutron Star Equation of State and Composition}














...

% \begin{figure}
% \centering
% \includegraphics[width=0.45\textwidth,angle=0]{TOVs.pdf}
% \caption{The mass-radius relationships calculated using the TOV equations for all of the $J$, $L$, $K_{\rm sym}$ values.}
% \label{fig:TOVs}
% \end{figure}








\section{Calculation of the Normal Modes}
\hspace{\parindent}Each mode of oscillation gives a different profile for the radial and transverse displacements. To calculate this, as well as the frequency the modes oscillate at, we perturb the equations defining the equilibrium state of neutron star matter.

\hspace{\parindent}The neutron stars involved in mergers are old and therefore will have cooled and spun down, so we assume the temperature is low throughout the star and ignore effects of rotation. We also assume that the periods of the modes of interest are significantly lower that the beta equilibrium timescale. Unless otherwise stated, the total masses of the neutron stars we have used are $1.4M_{\odot}$. Finally, we ignore terms that are non-linear in perturbations, as the perturbations are small and therefore non-linear terms are negligible.
% low temperature or zero temperature? is spun down correct?



%explain Eulerian vs Lagrangian perturbations?
\subsection{Linear Perturbation Equations}
\hspace{\parindent}Following the methodology of~\cite{mcdermott1988nonradial}, the equations governing the displacement caused by the modes of oscillation can be derived from the equations for mass continuity, momentum conservation, and Poisson's equation:
\begin{align}
\frac{\partial\rho}{\partial t}+\nabla\cdot(\rho v)=0,
\label{eq:continuity_eqn}
\end{align}
\begin{align}
\frac{\partial v}{\partial t}=(v\cdot \nabla)v=\frac{1}{\rho}\nabla\cdot\sigma-\nabla\Phi,
\label{eq:momentum_eqn}
\end{align}
\begin{align}
\nabla^2\Phi=4\pi G\rho,
\label{eq:Poisson_eqn}
\end{align}
\noindent where $\rho$ is the energy density, $v$ is the velocity of the matter, $\sigma$ is the stress tensor, and $\Phi$ is the gravitational potential. By adding a small perturbation to each of the quantities in these equations (eg: $\rho\rightarrow\rho+\delta\rho$), and ignoring terms that are non-linear in perturbations, these three equations can be used to obtain the wave equation
\begin{align}\nonumber
&&\omega^2u=-\nabla\left(\frac{\Gamma_1 p}{\rho}\nabla\cdot u\right)-\nabla\left(\frac{1}{\rho}u\cdot\nabla p\right)-\hat{r}A\frac{\Gamma_1 p}{\rho}\nabla\cdot u\\\nonumber
&&+\frac{1}{\rho}\biggr(\nabla\left(\frac{2}{3}\mu\nabla\cdot u\right)-\left(\nabla\mu\cdot\nabla\right)u-\nabla\left(u\cdot\nabla\mu\right)\\
&&+\left(u\cdot\nabla\right)\nabla\mu-\mu\left(\nabla^2 u+\nabla\left(\nabla\cdot u\right)\right)\biggr),
\label{eq:wave_eqn}
\end{align}
where $u(x,t)$ is the Lagrangian displacement (the displacement of the mass element initially located at $x$),
\begin{align}
A=\frac{1}{\rho}\frac{d\rho}{dr}-\frac{1}{\Gamma_1P}\frac{dP}{dr}
\label{eq:schwartz_descrim}
\end{align}
\noindent is the Schwarzschild discriminant, and
\begin{align}
\Gamma_1=\left(\frac{\partial\:ln(P)}{\partial\:ln(\rho)}\right)_s
\label{eq:adiabatic_index}
\end{align}
\noindent is the adiabatic index when the composition of the star is frozen. The shear modulus can be calculated using~\cite{strohmayer1991shear}
\begin{align}
\mu=\frac{0.1194}{1+17810\left(\frac{ak_bT}{\left(Ze\right)^2}\right)^2}\frac{n_i\left(Ze\right)^2}{a},
\label{eq:varymu}
\end{align}
\noindent where $T$ is the temperature, $n_i$ is the ion density, $Z$ is the proton number of the nuclei, and 
\begin{align}
a=\left(\frac{3}{4\pi n_i}\right)^{\frac{1}{3}}.
\label{eq:mu1991a}
\end{align}
\noindent We have also taken the Cowling approximation~\cite{cowling1941non}. This approximation states that, since the mass of a star is concentrated in its centre, the displacement of matter that is not close to the centre does not significantly alter the gravitational potential and therefore $\Phi '\approx 0$. The time dependence of the perturbed quantities was assumed to be of the form $e^{i\omega t}$, with $\omega$ being the frequency of the mode's oscillations. This can be used to separate the time and position dependencies of the Lagrangian displacement to get
\begin{align}
u(x,t)=\xi(x)e^{i\omega t}.
\label{eq:time_seperation}
\end{align}
\noindent In spherical coordinates this can be further separated into radial and transverse components:
\begin{align}
\xi_r=U(r)Y_{lm},\;\;\;\xi_{\theta}=V(r)\frac{\partial Y_{lm}}{\partial\theta},\;\;\;\xi_{\phi}=\frac{V(r)}{sin(\theta)}\frac{\partial Y_{lm}}{\partial\phi},
\label{eq:xi_seperation}
\end{align}
\noindent with $U(r)$ being the radial displacement, $V(r)$ being the transverse displacement, and $Y_{lm}$ being the spherical harmonics.

\hspace{\parindent}By using the separation of variables given in equations \ref{eq:time_seperation} and \ref{eq:xi_seperation}, equation \ref{eq:wave_eqn} becomes a set of 4 equations:% (or 5 without the Cowling approximation) 
\begin{align}\nonumber
&&\rho\omega^2U=\rho\frac{d\hat{\chi}}{dr}-A\Gamma_1 p\hat{\alpha}-\frac{d}{dr}\left(\frac{1}{3}\mu\hat{\alpha}\right)+\frac{d\mu}{dr}\left(\hat{\alpha}-2\frac{dU}{dr}\right)\\
&&-\mu\left(\frac{1}{r^2}\frac{d}{dr}\left( r^2\frac{dU}{dr}\right)-\frac{l(l+1)}{r^2}U+\frac{2l(l+1)}{r^2}V\right),
\label{eq:Ueqn}
\end{align}
\begin{align}\nonumber
&&\rho\omega^2V=\rho\frac{\hat{\chi}}{r}-\frac{1}{3}\frac{\mu\hat{\alpha}}{r}-\frac{d\mu}{dr}\left(\frac{dV}{dr}-\frac{V}{r}+\frac{U}{r}\right)\\
&&-\mu\left(\frac{1}{r^2}\frac{d}{dr}\left(r^2\frac{dV}{dr}\right)-\frac{l(l+1)}{r^2}V+\frac{2}{r^2}U\right),
\label{eq:Veqn}
\end{align}
% \begin{align}
% \frac{1}{r^2}\frac{d}{dr}\left(r^2\frac{d\hat{\Phi}'}{dr}\right)-\frac{l(l+1)}{r^2}\hat{\Phi}'=4\pi G\left(U\frac{d\rho}{dr}+\hat{\alpha}\rho\right),
% \label{eq:Phihat_eqn}
% \end{align}
\noindent where:
\begin{align}
\hat{\alpha}=\frac{1}{r^2}\frac{d}{dr}(r^2U)-\frac{l(l+1)}{r}V,
\label{eq:alphahat}
\end{align}
\begin{align}
\hat{\chi}=-\frac{\Gamma_1p}{\rho}\hat{\alpha}-\frac{1}{\rho}\frac{\partial p}{\partial r}U.
\label{eq:chihat}
\end{align}



\hspace{\parindent}For the fluid core of a neutron star, equations \ref{eq:Ueqn}-\ref{eq:chihat} are further simplified because in a fluid the shear modulus ($\mu$) must be zero. By using $\mu=0$, and be rewriting $U$ and $V$ as the dimensionless variables
\begin{align}
y_1=\frac{U}{r},\;\;\;\;y_2=\frac{\omega^2V}{g}
\label{eq:y1y2}
\end{align}
\noindent (where $g=\frac{1}{\rho}\frac{dP}{dr}$), equations \ref{eq:Ueqn}-\ref{eq:chihat} can be rewritten as a pair of coupled differential equations:
\begin{align}
\left(1+\tilde{V}\right)\frac{dy_1}{dx}=\left(\frac{\tilde{V}}{\Gamma_1}-3\right)y_1+\left(\frac{l(l+1)}{c_1\Omega^2}-\frac{\tilde{V}}{\Gamma_1}\right)y_2,
\label{eq:McDy1}
\end{align}
\begin{align}
\left(1+\tilde{V}\right)\frac{dy_2}{dx}=\left(c_1\Omega^2+Ar\right)y_1+\left(1-\tilde{U}-Ar\right)y_2,
\label{eq:McDy2}
\end{align}
\noindent where the equilibrium properties of the star are: 
\begin{align}\nonumber
\tilde{U}=\frac{d\:ln\left(M_r\right)}{d\:ln\left(r\right)}=4\pi r^2\rho\frac{r}{M_r},\;\;\;c_1=\left(\frac{r}{R_*}\right)^3\frac{M_*}{M_r},\;\;\;\Omega^2=\frac{\omega^2R_*^3}{GM_*},
\label{eq:eqbm_properties1}
\end{align}
\begin{align}\nonumber
\tilde{V}=-\frac{d\:ln\left(P\right)}{d\:ln\left(r\right)}=-\frac{G\rho M_r}{rP}\left(1+\frac{P}{\rho c^2}\right)\left(1+\frac{4\pi r^3 P}{M_r c^2}\right)\left(1-\frac{2GM_r}{rc^2}\right)^{-1},
\label{eq:eqbm_properties2}
\end{align}




\hspace{\parindent}For the solid crust the shear modulus is not zero, and therefore we obtain a set of four coupled differential equations:
\begin{align}
\left(1+\tilde{V}\right)\frac{dz_1}{dx}=-\left(1+2\frac{\alpha_2}{\alpha_3}\right)z_1+\frac{1}{\alpha_3}z_2+l\left(l+1\right)\frac{\alpha_2}{\alpha_3}z_3,
\label{eq:z1dr}
\end{align}
\begin{align}\nonumber
&&\left(1+\tilde{V}\right)\frac{dz_2}{dx}=\left(-\tilde{V}c_1\Omega^2-4\tilde{V}+\tilde{U}\tilde{V}+12\Gamma_1\frac{\alpha_1}{\alpha_3}\right)z_1\\\nonumber
&&+\left(\tilde{V}-4\frac{\alpha_1}{\alpha_3}\right)z_2+l\left(l+1\right)\left(\tilde{V}-6\Gamma_1\frac{\alpha_1}{\alpha_3}\right)z_3\\
&&+l\left(l+1\right)z_4,
\label{eq:z2dr}
\end{align}
\begin{align}
\left(1+\tilde{V}\right)\frac{dz_3}{dx}=-z_1+\frac{1}{\alpha_1}z_4,
\label{eq:z3dr}
\end{align}
\begin{align}\nonumber
&&\left(1+\tilde{V}\right)\frac{dz_4}{dx}= \biggr(-\tilde{V}c_1\Omega^2+\frac{2}{\alpha_3}\biggr((2l(l+1)-1)\alpha_1\alpha_2\\\nonumber
&&+2(l(l+1)-1)\alpha_1^2\biggr)\biggr)z_3\\
&&+\left(\tilde{V}-6\Gamma_1\frac{\alpha_1}{\alpha_3}\right)z_1-\frac{\alpha_2}{\alpha_3}z_2+(\tilde{V}-3)z_4,
\label{eq:z4dr}
\end{align}
\noindent where $\alpha_1=\frac{\mu}{\rho}$, $\alpha_2=\Gamma_1-\frac{2}{3}\frac{\mu}{\rho}$, $\alpha_2=\Gamma_1+\frac{4}{3}\frac{\mu}{\rho}$, and the dimensionless variables are 
\begin{align}
&z_1=\frac{U}{r},&z_2=\frac{1}{\rho}\left((\Gamma_1p-\frac{2}{3}\mu)\hat{\alpha}+2\mu\frac{dU}{dr}\right),\\
&z_3=\frac{V}{r},&z_4=\frac{\mu}{p}\left(\frac{dV}{dr}-\frac{V}{r}+\frac{U}{r}\right).
\label{eq:z1z2z3z4}
\end{align}
\noindent $z_1$ is the radial displacement, $z_2$ is proportional to the radial traction, $z_3$ is the transverse displacement, and $z_4$ is proportional to the transverse traction.



\hspace{\parindent}At the boundary between the solid crust and the fluid core (and the boundary between the fluid surface ocean and the solid crust), there are three jump conditions that must be satisfied:
\begin{align}
z_1=y_1,
\label{eq:jump1}
\end{align}
\begin{align}
z_2=\tilde{V}(y_1-y_2),
\label{eq:jump2}
\end{align}
\begin{align}
z_4=0.
\label{eq:jump3}
\end{align}
\noindent The first two of these conditions can be combined to obtain $\frac{z_2}{z_1}=\tilde{V}\left(1-\frac{y_2}{y_1}\right)$, resulting in two conditions and two eigenvalues, $\Omega$ and $z_3(R_*)$. 



\hspace{\parindent}For any mode the boundary conditions at the centre and surface of the star must be obeyed. The condition at the centre follows from the requirement that $y_1$ and $y_2$ be regular at $r=0$:
\begin{align}
\frac{c_1\Omega^2}{l}y_1-y_2=0.
\label{eq:core_condition}
\end{align}
\noindent The condition at the surface is based on the requirement that the Lagrangian pressure perturbation goes to zero at $r=R_*$ :
\begin{align}
\left(\tilde{V}-c_1\Omega^2-4+\tilde{U}\right)y_1+\left(\frac{l(l+1)}{c_1\Omega^2}-\tilde{V}\right)y_2=0,
\label{eq:surface_condition}
\end{align}
\noindent However, this boundary condition is for the surface ocean. The ocean is negligible for the interface modes, and so we skip directly to the crust. Therefore the surface boundary condition must be combined with the jump conditions (equations \ref{eq:jump1}-\ref{eq:jump3}) to rewrite it as
\begin{align}
z_2=\tilde{V}\left(\frac{\tilde{V}-c_1\Omega^2-4+\tilde{U}}{\frac{l(l+1)}{c_1\Omega^2}-\tilde{V}}+1\right)z_1,
\label{eq:surface_boundary_modified_1}
\end{align}
\begin{align}
z_4=0.
\label{eq:surface_boundary_modified_2}
\end{align}



\hspace{\parindent}By using trial values for the eigenvalues and using the shooting method to find the values of $y_1$ $y_2$, $z_{1-4}$ at the crust-core boundary, the modes can be found as the eigenvalues that satisfy the jump conditions. Figure \ref{fig:trace_minima} shows the eigenvalues at which each of the jump conditions are satisfied.

\begin{figure}
\centering
\includegraphics[width=0.45\textwidth,angle=0]{minima_new}
\caption{The zeros of the error in the jump conditions at the crust-core boundary, showing the location of two modes for the sly4 EoS (the interface and fundamental modes).}
\label{fig:trace_minima}
\end{figure}








\subsection{The Crust-Core Interface Mode}
\hspace{\parindent}The interface mode shown in figure \ref{fig:imode_sly4} has the radial displacement peaking at the crust-core boundary. For the sly4 EoS, its eigenvalues are $\Omega=0.1104$ and $z_3(R_*)=42.39$ (as can be seen in figure \ref{fig:trace_minima}), giving the mode a frequency of $188.5$ Hz. The transverse displacement in the core is small, while in the crust it discontinuously jumps to become large. This makes it a good candidate for a RSF, since having a higher displacement in the crust will result in most of the energy being in the crust.

\begin{figure}
\centering
\includegraphics[width=0.45\textwidth,angle=0]{sly4_imode}
\caption{The 2i mode for the sly4 equation of state. $V(r)$ has been reduced by an order of magnitude so that it can be plotted alongside $U(r)$.}
\label{fig:imode_sly4}
\end{figure}

\hspace{\parindent}The i-mode has a strong dependence on the shear modulus in the crust, and more specifically at the crust-core transition. We investigated this dependence by considering a shear modulus scaled by 
\begin{align}
\tilde{\mu} = B \mu,
\end{align}
\noindent where the constant $B$ was varied to determine its impact on the i-mode. The constant which controls the scaling due to temperature was not varied because for most of the crust $1>>17810\left(\frac{ak_bT}{\left(Ze\right)^2}\right)^2$, meaning that the temperature's impact on the shear modulus is small. Therefore this constant would need to be very large to change the shear modulus. Because of this only the results for varying $B$ are shown in figure \ref{fig:mu_B}. Modifying the shear modulus in this way is very artificial due to the shear modulus being strongly linked to the equilibrium properties of the star. Therefore, only changing the shear modulus should be seen as a rough guide to its impact on the frequency.

\begin{figure}
\centering
\includegraphics[width=0.45\textwidth,angle=0]{shear_mod_vs_freq_2}
\caption{The change in i-mode frequency caused by varying $\mu$. The red lines indicate the result for $B=1.0$.}
\label{fig:mu_B}
\end{figure}










\section{Results}
\hspace{\parindent}Figure \ref{fig:f_vs_Rcc_R*} shows the relationship between the frequency of the i-mode and both the location of the crust-core transition and the total radius of the star. This figure was created by varying the central pressure of the neutron star and using the TOV equations to find $R_*$ and $R_{cc}$. Then the frequency of the i-mode was calculated for these stars. Three different crust-core transition densities for the sly4 EoS were manually inputted to investigate its effect on the i-mode frequency, and from the figure it can been seen that the frequency does not always decrease with decreasing $n_{b,cc}$. For $n_{b,cc}=6.5\times10^{37}cm^{-3}$ there is a maximum i-mode frequency of $239 Hz$ at around $R_*=12.5 Km$, which places an upper limit on the frequency. However, the mass of a neutron star with this radius is very low ($0.45$ solar masses), and therefore the i-mode frequency would likely be significantly lower than this value. If this maximum appears for the EoSs parameteried by $J$, $L$ and $K_{\rm sym}$, we could use it to constrain $J$, $L$ and $K_{\rm sym}$ for stars whose mass is not known.

\begin{figure}
\centering
\includegraphics[width=0.45\textwidth,angle=0]{f_Rstar_Rcc_2}
\caption{The frequency of the i-mode for varying neutron star mass and radius. The figure shows both the total radius (solid lines) and crust-core radius (dashed lines) for each frequency. The stars indicate $1.4$ solar masses neutron stars. Increasing total radius causes the central pressure and total mass to decrease.}
\label{fig:f_vs_Rcc_R*}
\end{figure}




\subsection{Interface Mode dependence on Nuclear Parameters}
%1st line is repeated earlier in paper somewhere, is it too repetitive?
\hspace{\parindent}Several EoSs have been generated using a grid of $J$, $L$, and $K_{\rm sym}$ values. By using these EoSs as inputs to find the i-mode frequency and then interpolating between them, we obtained a set of frequency contours in the $J,L$ plane. This is shows in figure \ref{fig:freq_contours}, with the two graphs showing either end of a reasonable $K_{\rm sym}$ range. From this we see that changes in $L$ and $K_{\rm sym}$ have larger impacts on the the i-mode frequency than changes in $J$. This is what we would expect because the matter in neutron stars is highly asymmetric.



\begin{figure}
\centering
\includegraphics[width=0.45\textwidth,angle=0]{contours_20gap_K40}
\includegraphics[width=0.45\textwidth,angle=0]{contours_20gap_Km200}
\caption{Frequency contours in the $J,L$ plane, with the first figure being for $K_{\rm sym}=40$ MeV, and the second for $K_{\rm sym}=-200$ MeV. The gaps between the contours are $20$ Hz. The grid points indicate the results of the EoSs, between which the frequency was interpolated.}
\label{fig:freq_contours}
\end{figure}








\subsection{Constraining $J$, $L$, and $K_{\rm sym}$ with an RSF detection}
\hspace{\parindent}The timescale over which resonant excitation of a mode can occur is calculated as \cite{tsang2012resonant}
\begin{align}
t_{\rm res}\sim 8\times10^{-2}\text{s}\left(\frac{\mathcal{M}}{1.2\text{M}_{\odot}}\right)^{\frac{-5}{6}}\left(\frac{f_{\rm mode}}{100 \text{ Hz}}\right)^{\frac{-11}{6}},
\label{res_timescale}    
\end{align}
\noindent where $\mathcal{M}=\frac{M_1^{\frac{3}{5}}M_2^{\frac{3}{5}}}{(M_1+M_2)^{\frac{1}{5}}}$ is the chirp mass. This can be combined with the rate of change of the gravitational wave frequency
\begin{align}
\Dot{f}_{\rm gw}=\frac{f_{\rm gw}}{4.7\times10^{-3}\text{s}}\left(\frac{\mathcal{M}}{1.2\text{M}_{\odot}}\right)^{\frac{5}{3}}\left(\frac{f_{\rm gw}}{1000 \text{ Hz}}\right)^{\frac{8}{3}},
\label{res_timescale}    
\end{align}
\noindent where $f_{\rm gw}\approx f_{\rm mode}$. This gives a simple estimate of the range of frequencies over which resonance can occur:
\begin{align}
\delta f\sim t_{\rm res}\Dot{f}_{\rm gw}\sim17.02f_{\rm mode}\left(\frac{\mathcal{M}}{1.2\text{M}_{\odot}}\right)^{\frac{5}{6}}\left(\frac{f_{\rm mode}}{1000 \text{ Hz}}\right)^{\frac{8}{3}}\left(\frac{f_{\rm mode}}{100 \text{ Hz}}\right)^{\frac{-11}{6}}.
\label{res_timescale}    
\end{align}
\noindent From this we see that the resonance window gets larger as the frequency increases, with $\delta f$ scaling as $f_{\rm mode}^{\frac{11}{6}}$. For a resonance at $100$ Hz and a chirp mass of $1M_{\odot}$, $t_{\rm res}\sim 0.093$ s which gives a frequency range of $\delta f\sim 3.15$ Hz. This means that the spread of the contour in the $J,L$ (which is shown in figure \ref{fig:t_res_spread}) is small ($\delta L\approx 0.5-1.0$ MeV). Therefore the main uncertainty will come from the value of $K_{\rm sym}$, which causes a spread of up to $\sim 20$ MeV in $L$ at any given $J$ value (as can be seen in \ref{fig:freq_contours}).
%From this we see that the resonance window gets smaller as the frequency increases, giving a smaller range over which the resonance will be spread. For a resonance at $100$ Hz and a chirp mass of $1M_{\odot}$, $t_{\rm res}\sim 0.093$ s which gives a frequency range of $\delta f\sim 3.15$ Hz. This means that the spread of the contour in the $J,L$, which is shown in figure \ref{fig:t_res_spread} is small, and so the main uncertainty will come from the value of $K_{\rm sym}$, which causes a spread of up to $\sim 20$ MeV in $L$ at any given $J$ value.
%I've used t_res*df/dt to get $\delta$ f, which would give $\delta$ f proportional to f^(11/6). Is this correct?

\begin{figure}
\centering
\includegraphics[width=0.45\textwidth,angle=0]{tres_range_comp.pdf}
\caption{Frequency contours in the $J,L$ plane for $K_{\rm sym}$. The black line is for $f=100$ Hz, and the red lines are for $f=100+\delta f$ Hz. The green line is for $f=150$ Hz, and the blue lines are for $f=150+\delta f$ Hz.}
\label{fig:t_res_spread}
\end{figure}

\hspace{\parindent}Figure \ref{fig:M_vs_f} shows that the mass of the neutron star changes the frequency of the i-mode by up to 20\% from the frequency of a $1.4 M_{\odot}$ star, with the frequency decreasing as the mass of the star is increased. This means that the i-mode frequency will have some variation based on the uncertainty in the mass measurement of the star.% mass variation in J,L plane. 3 dimensions -> EoS,mass,frequency. This figure is for change in frequency due to mass as constant EoS, while later figure is change in EoS parameters due to mass at constant frequency.

\begin{figure}
\centering
\includegraphics[width=0.45\textwidth,angle=0]{M_vs_f}
\caption{The relationship between NS mass and the i-mode frequency for the sly4 EoS.}
\label{fig:M_vs_f}
\end{figure}

\hspace{\parindent}Figure \ref{fig:vary_mass_contours} is similar to figure \ref{fig:M_vs_f}, but where \ref{fig:M_vs_f} shows the change in frequency due to changes in the NS mass for a constant EoS, figure \ref{fig:vary_mass_contours} shows the change in the EoS parameters due to changes in the NS mass for a constant frequency. The mass values used were taken from GW170817 as an example of the range in NS mass measurements. The difference between the contours in the $L$ value at any given $J$ value is $1-1.5$ MeV. Improved measurements of NS mass would help to reduce this uncertainty.

\begin{figure}
\centering
\includegraphics[width=0.45\textwidth,angle=0]{K40_f150_Mcomp.pdf}
\caption{The effect of changing the total NS mass on the symmetry energy parameters that give a frequency of $150$ Hz.}
\label{fig:vary_mass_contours}
\end{figure}








\section{Discussion}
\hspace{\parindent}Far from symmetry we would expect higher order symmetry energy parameters ($L$, $K_{\rm sym}$) to have a stronger impact on the frequency than lower order terms ($J$). From figure \ref{fig:freq_contours} this does seem to be the case, with the contours covering a wide range of $J$ values and a smaller range of $L$ values. $K_{\rm sym}$ also has a significant impact on the i-mode frequencies, with the extremes of our $K_{\rm sym}$ range having a difference of $\Delta L\sim 20$ MeV for the same $J$ values. Figure \ref{fig:constraints} shows contours for supposed detections of RSFs at $100$ Hz and $150 Hz$, with the two contours for each frequency showing the range of $K_{\rm sym}$ values. $100-150$ Hz was chosen because the precursor occurs roughly half a second before the merger~\cite{zhong2019precursors}, and half a second before the merger, the frequency of GW170817 was $~100-150$ Hz~\cite{abbott2017merger}.

\hspace{\parindent}....................................







\subsection{Comparisons to other constraints}
\hspace{\parindent}There are many different constraints on the symmetry energy parameters that can be found from various properties of nuclei, such as the their masses~\cite{kortelainen2010nuclear} and neutron skins~\cite{chen2010density}. By plotting these constraints together, a small range can be found for the symmetry energy parameters $J$ and $L$~\cite{balantekin2014nuclear}. This range is shown in figure \ref{fig:constraints} alongside the contours for RSFs at $100$ Hz and $150 Hz$. %Cite Lattimer and mention that the dipole polarisability is wrong?
% on the graph of constraints in the J,L plane, my results are similar to those for dipole polarisability
% constrain K using the J,L values?

\begin{figure}
\centering
\includegraphics[width=0.45\textwidth,angle=0]{constraints_rough_2}
\caption{<Very rough, replace later> Range in $J,L$ due to the range of $K_{sym}$ values. The upper bound is for K=40 MeV, and the lower bound is for K=-200 MeV.}
\label{fig:constraints}
\end{figure}

\hspace{\parindent}From figure \ref{fig:constraints} we see that for an i-mode frequency of $100$ Hz, our constraint applies a similar restriction to the nuclear constraints, making this a useful angle from which to view the restrictions of the values of $J$ and $L$. However, for an i-mode frequency of $150$ Hz, the constraint is shifted to significantly lower L values. Therefore a detection of a precursor at $150$ Hz would improve the previous constraints on $J$ and $L$ to approximately $J\approx 30.75\pm 1.65$ MeV, $L\approx 48\pm 7$ MeV. In general, these results prefer $L$ values towards to lower end of the previous constrains. Figure \ref{fig:all_constraints} shows the combination of: the spread in frequency due to $t_{\rm res}$, the range in $L$ due to the uncertainty in stellar mass, and the spread in $L$ due to the range of $K_sym$ values used. This figure shows the most conservative estimate of the $J$, $L$ constraints, with the upper bounds using $M_*=1.17$ M$_{\odot}$, $f\rightarrow f-\delta f$ Hz, and $K_{sym}=40$ MeV and the lower bounds using $M_*=1.36$ M$_{\odot}$, $f\rightarrow f+\delta f$ Hz, and $K_{sym}=-200$ MeV.




%figure that combines all uncertainties (K_sym value, NS mass, t_res range), explain this one in text.
\begin{figure}
\centering
\includegraphics[width=0.45\textwidth,angle=0]{K_M_tres_ranges.pdf}
\caption{<Very rough, replace later> Combines the ranges from figures \ref{fig:t_res_spread}, \ref{fig:vary_mass_contours}, and \ref{fig:constraints}.} %Upper bound is K=40 MeV, lower bound is K=-200 MeV.
\label{fig:all_constraints}
\end{figure}















%%%%%%%%%%%%%%%%%%%% REFERENCES %%%%%%%%%%%%%%%%%%

% The best way to enter references is to use BibTeX:

\bibliographystyle{mnras}
\bibliography{example} % if your bibtex file is called example.bib


% Alternatively you could enter them by hand, like this:
% This method is tedious and prone to error if you have lots of references
%\begin{thebibliography}{99}
%\bibitem[\protect\citeauthoryear{Author}{2012}]{Author2012}
%Author A.~N., 2013, Journal of Improbable Astronomy, 1, 1
%\bibitem[\protect\citeauthoryear{Others}{2013}]{Others2013}
%Others S., 2012, Journal of Interesting Stuff, 17, 198
%\end{thebibliography}

%%%%%%%%%%%%%%%%%%%%%%%%%%%%%%%%%%%%%%%%%%%%%%%%%%

%%%%%%%%%%%%%%%%% APPENDICES %%%%%%%%%%%%%%%%%%%%%

\appendix

\section{Some extra material}

If you want to present additional material which would interrupt the flow of the main paper,
it can be placed in an Appendix which appears after the list of references.

%%%%%%%%%%%%%%%%%%%%%%%%%%%%%%%%%%%%%%%%%%%%%%%%%%


% Don't change these lines
\bsp	% typesetting comment
\label{lastpage}
\end{document}

% End of mnras_template.tex
